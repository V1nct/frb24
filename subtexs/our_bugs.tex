下面挑选我们的漏洞测试系统在ROS相关项目中检测出的几个典型漏洞进行分析,它们充分体现了本系统的价值。
在\href{https://github.com/ros-planning/navigation2/issues/4177}{Navigation2 Issue\#4177}中,
我们指出了一个典型的数组越界漏洞。

Navigation2 系统中的节点会监听 \texttt{\textbackslash costmap},来更新代价地图信息,
但是缺乏对代价地图信息合法性的检查。如伪代码\ref{lst:bug:msg}中的map消息,width和height字段
指出了代价地图的宽高,data字段给出了代价地图的数据。对于正常消息,$width\times height$ 
应该等于data数组的大小,但这一点对于传输错误或恶意投放的消息这并不是被严格保证的。

\begin{lstlisting}[language=, caption=漏洞\#4177恶意消息, label=lst:bug:msg]
ros2 topic pub /map nav_msgs/msg/OccupancyGrid "
...
info:
  ...
  width: 2
  height: 2
  ...
data: [-1, 0, 1, 1, -1]  " <- would cause buffer overflow
\end{lstlisting}

Navigation2 自动导航项目并没有对这一点进行检查,而直接用width和height声明存储data的数组变量。
当\texttt{width}和\texttt{height}大于数组\texttt{data}的实际大小时,会导致程序访问越界(SEGV)。
初始化地图时该bug可能触发,见\ref{lst:bug:segv}第6行;运行时,如果接收到外部更新的地图信息,
也可能触发该问题,见\ref{lst:bug:segv}第24行。

\begin{lstlisting}[language=C++, caption=漏洞\#4177问题代码, label=lst:bug:segv]
/* 初始化时可能发生 */
// create the costmap
costmap_ = new unsigned char[size_x_ * size_y_];

for (unsigned int it = 0; it < size_x_ * size_y_; it++) {
  data = map.data[it]; // <- SEGV
  if (data == nav2_util::OCC_GRID_UNKNOWN) {
    costmap_[it] = NO_INFORMATION;
  } else {
    ...
  }
}

/* 过程中也可能发生,比如接收到外部更新的地图信息 */
void CostmapSubscriber::toCostmap2D()
{
  auto msg = std::atomic_load(&costmap_msg_);
	...

  unsigned char * master_array = costmap_->getCharMap();
  unsigned int index = 0;
  for (unsigned int i = 0; i < msg->metadata.size_x; ++i) {
    for (unsigned int j = 0; j < msg->metadata.size_y; ++j) {
      master_array[index] = msg->data[index]; //<-SEGV
      ++index;
    }
  }
}
\end{lstlisting}

在Naviagtion2\href{https://github.com/ros-planning/navigation2/pull/3958}{PR\#3972}, 
\href{https://github.com/ros-planning/navigation2/pull/3958}{PR\#3958}, 
\href{https://github.com/ros-planning/navigation2/issues/3940}{Issue\#3940}中, 
我们报告了一个典型的并发bug,本系统通过插桩延迟技术,找到了Navigation2项目中Costmap结构数据竞争的根本原因,
删除了之前大量不必要的检查。

此bug发生在关闭期间,ROS系统使用Lifecycle状态机机制来管理节点生命周期,Controller、Planner、
CostmapROS都是具有生命周期的节点,CostmapROS是Controller和Planner的子节点(独立线程)。
当ROS2系统关闭时,Lifecycle状态机发送同一关闭信号,这三种节点都会直接对该信号进行响应然后执行资源释放程序,
此时三者关闭顺序是不一致且不确定的。

如果子线程CostmapROS节点已经自行释放,Controller或Planner实质失去了对其的控制权,
如果它们仍试图访问CostmapROS,就产生了重复释放(UAF)、资源冲突等一系列并发问题,
如伪代码\ref{lst:bug:issue3972}所示。

\begin{lstlisting}[language=C, caption=漏洞\#3972问题代码, label=lst:bug:issue3972]
// Planner or Controller:
Controller::on_cleanup(const rclcpp_lifecycle::State &){
	costmap_ros_->cleanup() // <-UAF, costmap already cleanup itself.
}
\end{lstlisting}

对于需要频繁开关的机器人节点,关闭期间内存漏洞可能产生严重后果,但往往被开发者忽略。在 Navigation2 
\href{https://github.com/ros-planning/navigation2/issues/4175}{Issue\#4175},
\href{https://github.com/ros-planning/navigation2/pull/4180}{PR\#4180},
\href{https://github.com/ros-planning/navigation2/issues/4166}{Issue\#4166},
\href{https://github.com/ros-planning/navigation2/pull/4176}{Pull\#4176},和 
\href{https://github.com/ros2/rclcpp/issues/2447}{RclCpp Issue\#2447} 中,
我们和ROS2开发者讨论了ROS2节点退出机制可能导致的一些并发问题。

ROS2惯例是将收到订阅消息、收到服务请求的回调函数交给执行器(Executor)执行,
因此执行器线程拥有该回调函数的函数指针,当该回调函数所属的类已被释放,
如伪代码\ref{lst:bug:issue4176}第5行,执行器仍可能访问该回调函数,造成UAF错误。

\begin{lstlisting}[language=C++, caption=漏洞\#4176问题代码, label=lst:bug:issue4176]
nav2_util::CallbackReturn
AmclNode::on_cleanup(const rclcpp_lifecycle::State & /*state*/){
  
  ...
  initial_pose_sub_.rest(); // <- UAF
  
  ...
  executor_thread_.reset();
}

\end{lstlisting}

