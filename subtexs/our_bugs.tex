相关讨论可见GitHub问题 \href{https://github.com/ros-planning/navigation2/issues/4177}{Issue\#4177},提出了ROS 2通信机制中的一个潜在的缓冲区溢出问题。

\begin{lstlisting}[language=]
ros2 topic pub /map nav_msgs/msg/OccupancyGrid "
header:
  stamp:
    sec: 0
    nanosec: 0
  frame_id: map
info:
  map_load_time:
    sec: 0
    nanosec: 0
  resolution: 0.05000000074505806
  width: 2
  height: 2
  origin:
    position:
      x: -10.0
      y: -10.0
      z: 0.0
    orientation:
      x: 0.0
      y: 0.0
      z: 0.0
      w: 1.0
data: [-1, ] " <- buffer overflow
\end{lstlisting}

当\texttt{size\_x\_}和\texttt{size\_y\_}大于数组\texttt{data}大小时,就会导致访问越界(SEGV)。初始化时和接收到\texttt{/map}消息时触发此问题,展示了危险的代码实践。

\begin{lstlisting}[language=cpp]
// create the costmap
costmap_ = new unsigned char[size_x_ * size_y_];

for (unsigned int it = 0; it < size_x_ * size_y_; it++) {
  data = map.data[it]; // <- SEGV
  if (data == nav2_util::OCC_GRID_UNKNOWN) {
    costmap_[it] = NO_INFORMATION;
  } else {
    ...
  }
}
\end{lstlisting}

相关问题和解决方案还包括GitHub \href{https://github.com/ros-planning/navigation2/pull/3958}{PR\#3972}, \href{https://github.com/ros-planning/navigation2/pull/3958}{PR\#3958}, \href{https://github.com/ros-planning/navigation2/issues/3940}{Issue\#3940}, 揭示了并发bug,并通过两个PullRequest得到解决,节省了大量不必要的检查。

最后,讨论了ROS 2节点退出机制可能出现的并发问题,详见 \href{https://github.com/ros-planning/navigation2/issues/4175}{Issue\#4175}, \href{https://github.com/ros-planning/navigation2/pull/4180}{PR\#4180}, \href{https://github.com/ros-planning/navigation2/issues/4166}{Issue\#4166}, \href{https://github.com/ros-planning/navigation2/pull/4176}{Pull\#4176}, 和 \href{https://github.com/ros2/rclcpp/issues/2447}{RclCpp Issue\#2447}。
